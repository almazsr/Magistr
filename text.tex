\documentclass[12pt]{article}
\usepackage[english, russian]{babel}
\usepackage{mathtext}
\usepackage[T2A]{fontenc}
\usepackage[utf8x]{inputenc}
\usepackage[a4paper,left=30mm,right=15mm,top=20mm,bottom=30mm,dvips]{geometry}
\usepackage{amsmath}
\usepackage{amssymb}
\usepackage[dvips]{graphicx,color}
\usepackage{psfrag}
\def\div{\mathop{\rm div}}
\renewcommand{\log}{\ln}
\makeatletter
\renewcommand{\@biblabel}[1]{#1.}
\makeatother
\renewcommand{\cot}{\ctg}
\def\const{\mathop{\rm const}}
\def\grad{\mathop{\rm grad}}
\def\Real{\mathop{\rm Re}}
\def\Imag{\mathop{\rm Im}}
\renewcommand{\theenumi}{\arabic{enumi}}
\renewcommand{\labelenumi}{\arabic{enumi}.}
\renewcommand{\theenumii}{.\arabic{enumii}}
\renewcommand{\labelenumii}{\arabic{enumi}.\arabic{enumii}.}
\renewcommand{\theenumiii}{.\arabic{enumiii}}
\renewcommand{\labelenumiii}{\arabic{enumi}.\arabic{enumii}.\arabic{enumiii}.}
\renewcommand{\baselinestretch}{1.3}
\begin{document}
\large
%
% --------------------------------------------------------------- ТИТУЛЬНЫЙ ЛИСТ --------------------------------------------------------
%
\thispagestyle{empty}
\vspace{1cm}
\begin{center}
\large
\textbf{
Министерство образования и науки Российской Федерации\\[0.5cm]
КАЗАНСКИЙ (ПРИВОЛЖСКИЙ) ФЕДЕРАЛЬНЫЙ УНИВЕРСИТЕТ\\[0.5cm]}
ИНСТИТУТ МЕХАНИКИ И МАТЕМАТИКИ\\[0.5cm]
КАФЕДРА АЭРОГИДРОМЕХАНИКИ\\[1cm]
\begin{tabular}{cl}
Специальность: & 010200 — механика\\
Специализация: & 010205 — механика жидкости, газа и плазмы\\[2cm]
\end{tabular}
\textbf{ВЫПУСКНАЯ КВАЛИФИКАЦИОННАЯ РАБОТА}\\
(Магистрская работа)\\
\textbf{РЕШЕНИЕ ЗАДАЧИ О ПУЛЬСИРУЮЩЕМ ТЕЧЕНИИ}\\[2cm]
        \begin{tabular}{lll}
        Работа завершена: & &\\
        20 июня 2014 г. \underline{\phantom{Четкая подписьЧеткая подпись}} (А.Р. Сираев) \\
        Работа допущена к защите: & &\\
        \begin{tabular}{l}
        Научный руководитель\\
        доктор технических наук, профессор
        \end{tabular} & &\\
        21 июня 2014 г. \underline{\phantom{Четкая подписьЧеткая подпись}} (Н.И. Михеев) \\
        \begin{tabular}{l}
        Заведующий кафедрой\\
        доктор физ.-мат. наук, профессор
        \end{tabular} & &\\
        24 июня 2014 г. \underline{\phantom{Четкая подписьЧеткая подпись}} (А.Г. Егоров) \\
        \end{tabular}
\vfill
        Казань -- 2014
\end{center}

\newpage

%
% --------------------------------------------------------------- СОДЕРЖАНИЕ --------------------------------------------------------
%
\tableofcontents

\newpage
\section*{Условные обозначения и сокращения}
\addcontentsline{toc}{section}{Условные обозначения и сокращения}
%
% --------------------------------------------------------------- УСЛОВНЫЕ ОБОЗНАЧЕНИЯ И СОКРАЩЕНИЯ --------------------------------------------------------
%
\begin{align*}
 \textrm{ОКЗА} & - \textrm{обратная краевая задача аэрогидродинамики} \\
  \textrm{ИНЖ} & - \textrm{идеальная несжимаемая жидкость} \\
  \textrm{ЭВМ} & - \textrm{электронно-вычислительная машина} \\
 \textrm{CUDA} & - \textrm{Compute Unified Device Architecture} \\
 & \textrm{(вычислительная унифицированная архитектура устройств)} \\
 \textrm{TPL} & - \textrm{Task Parallel Library (библиотека параллельных задач)} \\
 \textrm{CPU} & - \textrm{Central processing unit (центральное процессорное устройство)} \\
 \textrm{GPU} & - \textrm{Graphics processing unit (графическое процессорное устройство)} \\
 \textrm{GPGPU} & - \textrm{General-purpose graphics processing
 units (GPU общего назначения)} \\
 \textrm{API} & - \textrm{Application program interface}\\&  \textrm{(интерфейс прикладного программирования)} \\
 \textrm{.NET} & - \textrm{.NET Framework (программная платформа от
 Microsoft)} \\
 \textrm{PLINQ} & - \textrm{Parallel for Language Integrated Query }\\&  \textrm{(параллельность для интегрированного языка
 запросов)} \\
  M & - \textrm{исходное количество узлов на профиле} \\
  N & - \textrm{количество узлов на единичной окружности в канонической}\\&  \textrm{плоскости -- искомое число узлов на
  профиле} \\
\end{align*}

\newpage
%
% --------------------------------------------------------------- ВВЕДЕНИЕ --------------------------------------------------------
%
\section{Введение}
\newpage
\section{Постановка задачи}
Рассматривается нестационарное ламинарное течение вязкой несжимаемой жидкости по цилиндрической трубе круглого сечения радиуса $r_0$ и длины $x_0$ с заданным законом изменения перепада давления.

Требуется:
\begin{enumerate}
\item получить уравнения для пульсирующего ламинарного течения в круглой цилиндрической трубе в безразмерной форме;
\item получить численное решение полученных уравнений;
\item найти распределение скорости по радиусу трубы;
\item вычислить среднерасходную скорость, кинетическую энергию по периоду;
\item проанализировать полученные результаты, сделать выводы.
\end{enumerate}
\newpage
\section{Решение задачи}
\subsection{Уравнение нестационарного ламинарного движения вязкой жидкости в цилиндрической трубе круглого сечения}
Исходные уравнения были получены из уравнений Стокса:
\begin{equation}
\label{stocks_eq}
\begin{cases}
\cfrac{\partial u}{\partial t}+u\cfrac{\partial u}{\partial x}+v\cfrac{\partial u}{\partial y}+w\cfrac{\partial u}{\partial z}=F_x-\cfrac{1}{\rho}\cfrac{\partial p}{\partial x}+\nu\nabla^2u,\\
\cfrac{\partial v}{\partial t}+u\cfrac{\partial v}{\partial x}+v\cfrac{\partial v}{\partial y}+w\cfrac{\partial v}{\partial z}=F_y-\cfrac{1}{\rho}\cfrac{\partial p}{\partial y}+\nu\nabla^2v,\\
\cfrac{\partial w}{\partial t}+u\cfrac{\partial w}{\partial x}+v\cfrac{\partial w}{\partial y}+w\cfrac{\partial w}{\partial z}=F_z-\cfrac{1}{\rho}\cfrac{\partial p}{\partial z}+\nu\nabla^2w\\
\cfrac{\partial u}{\partial x}+\cfrac{\partial v}{\partial y}+\cfrac{\partial w}{\partial z}=0.
\end{cases}
\end{equation}
В рамках данной задачи ось трубы совпадает с осью $x$: $v=w=0$, объемные силы отсутствуют: $F_x=F_y=F_z=0$. При учете уравнения неразрывности получено, что $\cfrac{\partial u}{\partial x}=0$. Третье уравнение из системы уравнения Стокса примет вид
\begin{equation}
\label{tube_eq}
\cfrac{\partial u}{\partial t}-\nu\nabla^2u=-\cfrac{1}{\rho}\cfrac{\partial p}{\partial x}.
\end{equation}
Левая часть уравнения зависит только от $y$, $z$ и $t$.
При этом, из первых двух уравнений \eqref{stocks_eq} следует, что $\cfrac{\partial p}{\partial y}=\cfrac{\partial p}{\partial z}=0$
Таким образом, правая часть уравнения не зависит от $y$ и $z$. Это означает, что $\cfrac{\partial p}{\partial x}$ является функцией зависящей только от времени
\begin{equation}
\label{pres_grad_eq}
-\cfrac{\partial p}{\partial x}=\rho f(t).
\end{equation}
При переходе в цилиндрическую систему координат и подстановке в уравнение  \eqref{tube_eq} выражения для градиента давления \eqref{pres_grad_eq} получено уравнение нестационарного ламинарного движения вязкой жидкости в цилиндрической трубе круглого сечения
\begin{equation}
\label{tube_cylindric_eq}
\cfrac{\partial u}{\partial t}-\nu\left(\cfrac{\partial^2u}{\partial r^2}+\cfrac{1}{r}\cfrac{\partial u}{\partial r}\right)=f(t)
\end{equation}
\subsection{Переход к безразмерной форме}
Закон изменения перепада давления задан следующим образом:
\begin{equation}
\label{pres_grad_explicit_form}
\cfrac{\partial p}{\partial x}=\rho \left(A\cos\omega t+\cfrac{\Delta p}{\rho x_0}\right)
\end{equation}
Связь между размерными и безразмерными переменными:
\begin{equation}
\begin{array}{lllllllll}
t=t_0\bar{t},&
u=u_0 \bar{u},&
r=r_0 \bar{r},&
x=x_0 \bar{x}
\end{array}
\end{equation}
Выбраны следующие масштабы:
\begin{equation}
\label{mashtabi} 
\begin{array}{lllll}
t_0=\cfrac{1}{\omega}&,&
u_0=\sqrt{\cfrac{\Delta p}{\rho}}&,&
u'=\sqrt{\strut{A r_0}}
\end{array}
\end{equation}
После подстановки \eqref{pres_grad_explicit_form} и введеных масштабов \eqref{mashtabi} уравнение \eqref{tube_cylindric_eq} примет вид
\begin{equation}
\omega u_0\cfrac{\partial \bar{u}}{\partial \bar{t}}-\nu \cfrac{u_0}{r_0^2}\left(\cfrac{\partial^2\bar{u}}{\partial \bar{r}^2}+\cfrac{1}{\bar{r}}\cfrac{\partial \bar{u}}{\partial \bar{r}}\right)=\cfrac{u'^2}{r_0}\cos\bar{t}+\cfrac{u_0^2}{x_0}
\end{equation}
Обе части уравнения умножены на $\cfrac{r_0^2}{\nu u_0}$ (черта над безразмерными переменными опущена)
\begin{equation}
\label{bezrazmer_eq_vivod}
\omega \cfrac{r_0^2}{\nu}\cfrac{\partial u}{\partial t}-\left(\cfrac{\partial^2u}{\partial r^2}+\cfrac{1}{r}\cfrac{\partial u}{\partial r}\right)=\cfrac{u_0 r_0 u'^2}{\nu u_0^2}\cos t+\cfrac{u_0 r_0^2}{\nu x_0}
\end{equation}
После введения безразмерных комплексов
\begin{equation*}
\begin{array}{lllllll}
s=r_0\sqrt{\cfrac{\omega}{\nu}}&,&
Re=\cfrac{u_0 r_0}{\nu}&,&
\beta=\cfrac{u'}{u_0}&,&
\varepsilon=\cfrac{r_0}{x_0}
\end{array}
\end{equation*}
уравнение \eqref{bezrazmer_eq_vivod} примет окончательный вид в безразмерной форме:
\begin{equation}
\label{bezrazmer_eq}
s^2\cfrac{\partial u}{\partial t}-\left(\cfrac{\partial^2u}{\partial r^2}+\cfrac{1}{r}\cfrac{\partial u}{\partial r}\right)=Re\left(\beta^2\cos t+\varepsilon\right).
\end{equation}
Также, удобным для расчетов будет другой вид уравнения \eqref{bezrazmer_eq}, в котором используются комплексы
\begin{equation*}
\begin{array}{lllllll}
H_1=s^2&,&
H_2=Re\beta^2&,&
H_3=Re\varepsilon&,&
\end{array}
\end{equation*}
которые полностью определяют решение. 
Тогда
\begin{equation}
\label{bezrazmer_eq_withH}
H_1\cfrac{\partial u}{\partial t}-\left(\cfrac{\partial^2u}{\partial r^2}+\cfrac{1}{r}\cfrac{\partial u}{\partial r}\right)=H_2\cos t+H_3.
\end{equation}
\subsection{Точное решение в случае осцилирующего ламинарного течения}
\newpage
\section{Примененные алгоритмы при решении задачи}
\subsection{Формулы для вычисления функций Бесселя}
\subsection{Разностные схемы для уравнения диффузии в цилиндрических координатах}
\subsection{Сравнение результатов точного и численного решения}
\subsection{Зависимость сходимости по периоду от численной схемы}
\subsection{Выбор разностной схемы для проведения анализа решения}
\newpage
\section{Анализ полученного решения}
\subsection{Зависимость профиля скорости от варьируемых комплексов}
\subsection{Крайние случаи}
\subsection{Анализ среднерасходной скорости}
\newpage
\section{Заключение}
\section{Список использованной литературы}
%
% --------------------------------------------------------------- ВВЕДЕНИЕ --------------------------------------------------------
%
\end{document}
